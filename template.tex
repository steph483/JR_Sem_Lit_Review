\documentclass[11pt,twocolumn]{article} 

% required packages for Oxy Comps style
\usepackage{oxycomps} % the main oxycomps style file
\usepackage{times} % use Times as the default font
\usepackage[style=numeric,sorting=nyt]{biblatex} % format the bibliography nicely

\usepackage{amsfonts} % provides many math symbols/fonts
\usepackage{listings} % provides the lstlisting environment
\usepackage{amssymb} % provides many math symbols/fonts
\usepackage{graphicx} % allows insertion of grpahics
\usepackage{hyperref} % creates links within the page and to URLs
\usepackage{url} % formats URLs properly
\usepackage{verbatim} % provides the comment environment
\usepackage{xpatch} % used to patch \textcite

\bibliography{references}
\DeclareNameAlias{default}{last-first}

\xpatchbibmacro{textcite}
  {\printnames{labelname}}
  {\printnames{labelname} (\printfield{year})}
  {}
  {}

\pdfinfo{
    /Title (Experiencing Mexican Culture through cuisine in VR)
    /Author (Stephanie Enriquez Isais)
}

\title{Experiencing Mexican Culture through cuisine in VR}

\author{Stephanie Enriquez Isais}
\affiliation{Occidental College}
\email{senriquezisa@oxy.edu}

\begin{document}

\maketitle

\begin{abstract}
    This literature review gives an insight into the problem context, technical background and prior work I considered when choosing my senior comprehensive project. I decided to do a virtual reality (VR) project that allows users to experiences Mexican cuisine through a historical farm-to-table view of it's preparation. 
\end{abstract}

\section{Problem Context}
There are 69 official languages in Mexico and each of it’s 32 states has its own distinct traditions, style and cuisine. From an outsider’s perspective, Mexican culture is often just associated with sombreros, tacos, burritos, and tequila, and while these do make up a part of Mexican culture, they are nowhere near representative of all Mexico has to offer. Mexican cuisine is extremely diverse and has a lot of historical significance. There are drinks, like tepache (fermented pineapple juice), that have pre-colonial roots and are still widely enjoyed by people in Mexico. However, unless you live in Mexico or have some Mexican heritage you probably won’t ever encounter these parts of Mexican culture. In addition to that, the way Mexican food was prepared in the past is not commonly used in the average Mexican household anymore. Tools like the metate (mealing stone) were used by Mexican people in the past to grind their corn or other ingredients into a fine powder. The word metate comes from the Nauhuatl word metlatl which means grinding stone. The indigenous Mexican people of the past used metates to grind the maize (corn) down and be able to make tortillas. These tools and techniques have historical value to Mexican culture because they influenced the style of food made and should not be forgotten . Even when talking to my Mexican American friends, most of them don’t know what a metate is, but their grandparents or even parents probably do know what it is and how it was used. Although most Mexican households have replaced the metate with a food processor or a molcajete (pestle and mortar) there are a couple of people who still use them. My dad tells me about the metate my grandmother was given when she got married and the pride she took in how much use she got out of it and the unique taste it gave her food. This is the historical context that I want to teach people about. It is a bit unfortunate that VR can only provide a visual and audio experience because what often distinguishes these traditional tools from current ones is the special flavor they add to the food. Although VR does not provide a way to showcase the unique taste and labor that goes into using these Mexican tools, I still think it’s a great way of getting people interested in the historical importance of them and have the next best thing to actually using the tools and making the food. These traditions have shaped Mexican culture and offer Mexicans a way of staying connected to our indigenous roots. 
 

\section{Technical Background}
HUD: Although my project is not a game, it does require the usage of game design elements. One of them being the HUD (FULL NAME). This UI element allows the user to have easy see where they stand in the game, whether this means their literal position or their health/remaining track. While in traditional console, PC or mobile games the HUB is a constant interface that can be seen on the outer edges of the screen, in VR this is usually not the case. 

Curve Screen: FIXME
 
Hand tracking: FIXME



\section{Prior Work}
When doing research into the prior work done to address this problem, I found a couple of VR apps that are related to my VR project idea. The first one is called Lost Recipes, this VR game teaches users to cook by showing an array of traditional dishes from ancient cultures like the Mayan, Greek and Chinese \cite{lostrecipes2022}. The game guides the players through tasks like measuring and mixing ingredients in an effort to recreate a traditional recipe. While this game does a good job of encouraging the users to complete the tasks and even try the recipes out on their own (this is based on some reviews left on the game), it does not give much historical information about the significance of the food or tools they are using. Since this is the game that is most similar to my idea, it has helped me identify the areas that I would like to improve on. At the same time, playing the game has taught me about the manner in which affordances can be used to have the user complete the game in the intended manner and gain the most out of it. Another similar cooking game is Cooking Simulator \cite{cookingsim2019}. As the name suggests this game is a lot more focused on creating a realistic experience of what it is like to cook in VR. The user has to measure each ingredient, can cut the produce and can even cook or bake in the game. This game is probably one of the most realistic cooking simulators and gives good examples of how to program the tools in the kitchen so the user can use them with controllers. When looking at references for how to build my cooking mechanics I think this game will be a great example to look back upon. In addition to the cooking aspect of my project, I want to really take advantage of the immersiveness of VR and create a VR experience that can tell` a story in a more engaging way. The first example is the VR game “Vader Immortal” \cite{vadarimmortal2019ep1}. The game revolves around the player trying to stop Darth Vader from completely destroying the planet Mustafar. When presenting the player with the history of the planet and why they should fight for it, they do it through the use of 3D animations. The majority of the game is the user traveling, fighting, and exploring the world but when presenting what is at stake, the creators decided to use storytelling through animation and after playing the game, it is clear why they made this choice. Seeing the story quite literally unravel around you makes it more engaging and immersive. It inspires me to create a similar sentiment when describing the history of Mexican cuisine. The Vader immortal approach to storytelling is similar to the cut scenes that are often used in video games but another technique for storytelling that would be interesting is the one the VR game “What Remains of Edith Finch” \cite{finch2019}. This game details the stories about what happened to the Finch family and one scene does this very interesting of incorporating the mundane chore of cutting fish heads in an assembly line with character’s daydreams slowly taking up more and more of the screen space. I think incorporating something similar to this in which the user is using one of the labor intensive cooking tools and then they see a scene of the history of the ingredients they are using would be an interesting way of engaging the user with the content I am trying to teach them about. 

\printbibliography 

\end{document}
